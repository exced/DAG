\documentclass{report}

\usepackage[utf8]{inputenc}
\usepackage[T1]{fontenc}
\usepackage[francais]{babel}
\usepackage[top=2.5cm, bottom=2.5cm, left=2.5cm, right=2.5cm]{geometry}
\usepackage{graphicx}
\usepackage{amsmath,amsfonts,amssymb}
\usepackage{parallel}
\usepackage{float}
\restylefloat{table}

\usepackage[linesnumbered,algoruled,boxed,lined]{algorithm2e}

\setlength\parindent{24pt} % indentation paragraphes

\begin{document}
\begin{titlepage}

\newcommand{\HRule}{\rule{\linewidth}{0.5mm}} % Defines a new command for the horizontal lines, change thickness here

\center % Center everything on the page
 
%----------------------------------------------------------------------------------------
%	HEADING SECTIONS
%----------------------------------------------------------------------------------------

\textsc{\LARGE ENSEEIHT}\\[1.5cm] % Name of your university/college
\textsc{\Large Graphes}\\[0.5cm] % Major heading such as course name
\textsc{\large Projet}\\[0.5cm] % Minor heading such as course title

%----------------------------------------------------------------------------------------
%	TITLE SECTION
%----------------------------------------------------------------------------------------

\HRule \\[0.4cm]
{ \huge \bfseries  Ordonnancement parallèle de tâches sur un graphe orienté acyclique}\\[0.4cm] % Title of your document
\HRule \\[1.5cm]
 
%----------------------------------------------------------------------------------------
%	AUTHOR SECTION
%----------------------------------------------------------------------------------------

\emph{auteurs:}\\
\textsc{barras} - \textsc{thomas} \\
\textsc{maurel} - \textsc{simon} \\

%----------------------------------------------------------------------------------------
%	DATE SECTION
%----------------------------------------------------------------------------------------

\date{} %enleve la date

%----------------------------------------------------------------------------------------
%	LOGO SECTION
%----------------------------------------------------------------------------------------

\vspace{7.2cm}
\begin{center}
%\includegraphics[width=5]{N7.png}
\end{center}

%----------------------------------------------------------------------------------------

\vfill % Fill the rest of the page with whitespace

\end{titlepage}

\tableofcontents %table des matières.

\vfill

\section{Question 1}

\begin{itemize}
	\item[(a)] Les noeuds du DAG représentent l'existence d'un sommet du maillage. 
	\item[(b)] La condition "un noeud v1 doit être traité avant un noeud v2" est : $\E$ un chemin reliant v2 et v1 qui respecte l'orientation.
\end{itemize}


\section{Question 2}

\begin{itemize}
	\item[(a)] Pour inclure le coût des communications, on pondère les arêtes.
	\item[(b)] L'équilibrage de charges se traduit par le fait que l'arbre a une profondeur faible. \\
	Le volume de communications réduit se traduit par le fait que la pondération totale des arcs reliant 2 partitions soit minimale. 
	\item[(c)] 
\end{itemize}

\section{Question 3}

\begin{itemize}
	\item[]
\end{itemize}



\end{document}